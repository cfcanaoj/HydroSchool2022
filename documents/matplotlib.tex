%------------------------------------------------------------
\chapter{ Pythonによるデータの可視化 (動画作成)}
%------------------------------------------------------------


Pythonはスクリプト言語の一種で,各種ライブラリが充実しているため,
シミュレーション結果の解析や可視化を手軽に行うことができ,天文学の分野でも広く使われている。
この節ではデータ解析でよく用いられる{\ttfamily Numpy} \citep{Numpy}と,可視化ライブラリである{\ttfamily Matplotlib} \citep{Matplotlib}
の使い方を簡単に説明する。

\begin{itemize}
    \item {\ttfamily Numpy} \\
        Pythonの処理速度は,FortranやC言語に比べて非常に遅く,Pythonの配列に当たるリストやタプルを使って
データ解析をおこなうことは困難である。
{\ttfamily Numpy}は,内部でFortranやC言語で実装されているため,処理が遅いという欠点を克服する,配列処理に特化したライブラリである。

\item {\ttfamily Matplotlib} \\
    Python用の可視化ツール。昨今では多くの論文の図が{\ttfamily Matplotlib}を使って作成されている。
\end{itemize}

ここでは,サンプルプログラムを元に基本的な使い方を解説するが,{\ttfamily numpy}と{\ttfamily matplotlib}の機能を全て網羅できないし,
私自身一部の機能しか知らないので,詳しくは解説書などを参照してほしい。


\section{環境整備}

CfCAで提供している{\ttfamily virtualbox}にすでに{\ttfamily anaconda3}がインストールされている。

\section{1次元データの解析と可視化}

移流方程式を使ったシミュレーション結果の解析用サンプルプログラム「{\ttfamily adv\_data\_analysis.py}」を提供している。
$U$の空間分布のスナップショット名は「{\ttfamily adv?????.dat}」である。「?」には数字が入り,連番となっている。

\lstinputlisting[]{adv_plot.py}

\begin{itemize}
    \item 1行目:\fbox{\ttfamily import numpy as np}は,{\ttfamily numpy}をロードして,{\ttfamily np}と呼びますよ,という意味。
    \item 4行目:\fbox{\ttfamily for istep in range(100):}は,{\ttfamily istep}を0から99までループする。
        Fortranで言うと\fbox{\ttfamily do istep=0,99},
        C言語で言うと\fbox{\ttfamily for(istep=0; istep<100; istep++)},

\end{itemize}


\section{2次元データの解析と可視化}



